
\chapter{Estructura de la memoria}

\lettrine{L}{a} memoria se estructurará en varios capítulos detallados a continuación:

\begin{itemize}
	\item \textbf{Introducción:} En este capítulo explicaremos la motivación y los objetivos del estudio, además de presentar la estructura de la memoria.
	\item \textbf{Estado del Arte:} Se definirán términos relativos al dominio del problema y se realizará un breve análisis de las investigaciones previas sobre el análisis de sentimientos.
	\item \textbf{Fundamentos tecnológicos:} Capítulo en el que se describirán las tecnologías y herramientas utilizadas para la implementación del proyecto.
	\item \textbf{Metodología de desarrollo:} Se explicará la metodología utilizada.
	\item \textbf{Planificación:} Se detallará la planificación llevada a cabo para afrontar el proyecto.
	\item \textbf{Análisis:} Definiremos las arquitecturas seleccionadas para el proyecto, comenzando por una definición de la arquitectura general y posteriormente detallando las arquitecturas propias de cada subsección del proyecto.
	\item \textbf{Casos de uso:} Apartado en el que se detallarán los casos de uso a abordar en la implementación del proyecto.
	\item \textbf{Implementación:} En este capítulo explicaremos la puesta en marcha del proyecto, explicando los detalles de la implementación tanto para el problema de clasificación como para el desarrollo del servicio web.
	\item \textbf{Resultados:} Análisis de los resultados obtenidos en los distintos experimentos.
	\item \textbf{Conclusiones:} Se tratará de llegar a una serie de conclusiones a partir de los resultados presentados previamente.
	\item \textbf{Solución desarrollada:} Apartado en el que se mostrará el resultado final del desarrollo.
\end{itemize}